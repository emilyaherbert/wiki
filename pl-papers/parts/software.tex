\chapter {Software}

\section{Timeline}

\scalebox{1}{
\begin{tabular}{r |@{\foo} l}

2018 & Calculation View: multiple-representation editing in spreadsheets \cite{sarkar2018calculation} \\
     & Elastic Sheet-Defined Functions: Generalising Spreadsheet Functions to Variable-Size Input Arrays \cite{elastic-sheet-defined-functions-generalising-spreadsheet-functions-to-variable-size-input-arrays} \\

\end{tabular}
}

\section {Calculation View: multiple-representation editing in spreadsheets
\cite{sarkar2018calculation}}
Presents "Calculation View" (CV), a novel alternate view in Microsoft Excel.
Allows users to edit spreadsheets in a more high-level way. Introduces ranges,
named ranges, pseudocells, and a block detection algorithm. Details further work
to expand CV view, including text interface and expanded features.
\begin{itemize}
    \item \question{Can CV store intermediate values that the user might not need represented on the sheet?}
    \item \question{Can CV be used on its own? Is there a use case for CV being used on its own?}
\end{itemize}

\section {Elastic Sheet-Defined Functions: Generalising Spreadsheet Functions to
Variable-Size Input Arrays \cite{elastic-sheet-defined-functions-generalising-spreadsheet-functions-to-variable-size-input-arrays}}
Defines formal syntax and semantics for generalising spreadsheet functions to
variable-size input arrays (ranges). Outlines algorithm for identifying
generalizations and interpreting most likely generalizations given multiple
options.