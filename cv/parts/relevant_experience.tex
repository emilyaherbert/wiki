\section{Relevant Experience}

\cventry{Aug 2021 - present}{Fuel Labs}%
{}%
{}%
{}%
{Compiler Engineer - Rust
\newline{Designed and implemented a compiler for the Sway smart contract
programing language, focusing on friendly interactivity with developers.
Built tools to integrate with the FuelVM and the Fuel ecosystem.}
%\newline{}
}

\cventry{}{Talks \& Panels}{}{}{}{}
\small
\cvitem{}{
Blockchain Language Design panel.
\textit{Layer 2 Day at EthDenver.} 2023.
[\href{https://youtu.be/cH4BNwwZLv0?t=74}{{\color{navyblue}recording}}]
%\newline{}
}
\cvitem{}{
Introduction to Fuel: Let's Get Modular talk.
\textit{Celestia's Modular Fellows.} 2022.
[\href{https://emilyaherbert.github.io/wiki/talks/2022/ModularFellows/}{{\color{navyblue}slides}}]
%\newline{}
}
\cvitem{}{
Beyond Monolithic with Fuel: the Fastest Modular Execution Layer talk.
\textit{HackMoney.} 2022.
[\href{https://emilyaherbert.github.io/wiki/talks/2022/HackMoney/}{{\color{navyblue}slides}}]
%\newline{}
}
\cvitem{}{
Scaling Execution: Optimistic panel.
\textit{The Modular Summit.} 2022.
%\newline{}
}
\cvitem{}{
Developing Smart Contracts in Sway talk talk.
\textit{Layer2Amsterdam.} 2022.
[\href{https://emilyaherbert.github.io/wiki/talks/2022/Layer2Amsterdam/slides.pdf}{{\color{navyblue}slides}}]
%\newline{}
}
\cvitem{}{
Sway: A Rust-based Smart Contract Language talk.
\textit{EthDenver.} 2022.
[\href{https://emilyaherbert.github.io/wiki/talks/2022/EthDenver/}{{\color{navyblue}slides}}]
%\newline{}
}
\cvitem{}{
The Rollup Developer Experience panel.
\textit{EthDenver.} 2022.
[\href{https://www.youtube.com/watch?v=Tgk7eXUCgYk&t=1245s}{{\color{navyblue}recording}}]
\newline{}
}
\normalsize

\cventry{May 2021 – Aug 2021}{Google}%
{Madison, WI}%
{}%
{}%
{Software Engineering Intern
\newline{Implemented load balancing in a library meant to interface with the
network card and perform RPC-like operations using RMA, achieved by integrating
two existing early-development libraries together. Contributed to app design on
a Google 2023 project.}
\newline{}
}

\cventry{June 2018 - May 2021}{Northeastern University, University of Massachusetts Amherst}%
{Boston, MA, Amherst, MA}%
{}%
{}%
{Researcher
\newline{Researching programming language and systems tools for serverless computing. %
\href{https://prl.ccs.neu.edu/}{{\color{navyblue}prl.ccs.neu.edu}}. %
\href{https://plasma-umass.org/}{{\color{navyblue}plasma-umass.org}}.}
%\newline{}
}

\cventry{}{Talks \& Publications}{}{}{}{}
\small
\cvitem{}{
Emily Herbert and Arjun Guha. A Language-based Serverless Function
Accelerator. 2021.
[\href{https://arxiv.org/abs/1911.02178}{{\color{navyblue}preprint}},
\href{https://github.com/emilyaherbert/containerless}{{\color{navyblue}repo}}]
%\newline{}
}
\cvitem{}{
A Language-based Serverless Function Accelerator talk.
\textit{Google PhD Intern Research Conference.} 2021.
%\newline{}
}
\cvitem{}{
A Language-based Serverless Function Accelerator talk.
\textit{Cornell CAPRA Lab.} 2020.
[\href{https://emilyaherbert.github.io/wiki/talks/2020/cornell-capra-lab.pdf}{{\color{navyblue}slides}}]
\newline{}
}
\normalsize

\cventry{June 2018 - May 2019}{University of Massachusetts Amherst}%
{Amherst, MA}%
{}%
{}%
{Researcher
\newline{Researching deep learning methods for simulation input modeling. %
\href{http://dbgroup.cs.umass.edu/}{{\color{navyblue}dbgroup.cs.umass.edu}}}
%\newline{}
}

\cventry{}{Talks \& Publications}{}{}{}{}
\small
\cvitem{}{
Wang Cen, Emily A. Herbert, and Peter J. Haas. NIM: Modeling and
Generation of Simulation Inputs via Generative Neural Networks. \textit{Winter
Simulation Conference.} 2020.
[\href{https://emilyaherbert.github.io/wiki/papers/con177s3.pdf}{{\color{navyblue}paper}}] \textbf{Best Contributed Theoretical Paper Finalist}
%\newline{}
}
\cvitem{}{
Wang Cen, Emily A Herbert, Peter J Haas. Generative Neural Networks for Simulation Input Modeling.
\textit{SCS Summer Simulation Conference.} 2019.
[\href{https://emilyaherbert.github.io/wiki/talks/2019/326.pdf}{{\color{navyblue}extended abstract}}]
%\newline{}
}
\cvitem{}{
Emily A Herbert, Wang Cen, and Peter J Haas. NIM: Generative Neural Networks for Simulation Input Modeling.
\textit{SCS Summer Simulation Conference.} 2019.
[\href{https://emilyaherbert.github.io/wiki/papers/summersim19.pdf}{{\color{navyblue}short paper}}, \href{https://emilyaherbert.github.io/wiki/talks/2019/summer-sim.pdf}{{\color{navyblue}slides}}]
\newline{}
}
\normalsize

\cventry{June 2017 – Aug 2017}{National Aeronautics and Space Administration
(NASA)}%
{Langley, VA}%
{}%
{}%
{NASA Internships, Fellowships, and Scholarships (NIFS) Intern
\newline{Contributed to the NASA Safeguard autonomous drone geofencing project.
Designed and implemented system for on-board flight control of GPS devices.
Refactored code from previous NASA flight missions to meet current mission
standards.}%
\newline{}
}

%\cventry{June 2016 – Aug 2016}{General Electric (GE), Oil \& Gas}%
%{Billerica, MA}%
%{}%
%{}%
%{Information Technology Leadership Program (ITLP) Intern
%\newline{Created asset tracking system for shop floor using RFID, Bluetooth LE,
%and Raspberry Pi. Worked with the SAP enterprise resource management software to
%automate EHSM compliance checks.}%
%%\newline{}
%}